%% LyX 2.0.3 created this file.  For more info, see http://www.lyx.org/.
%% Do not edit unless you really know what you are doing.
\documentclass[english]{article}
\usepackage{mathptmx}
\usepackage{helvet}
\usepackage{courier}
\usepackage[T1]{fontenc}
\usepackage[latin9]{inputenc}
\usepackage{listings}
\usepackage{geometry}
\geometry{verbose,tmargin=2.54cm,bmargin=2.54cm,lmargin=2.54cm,rmargin=2.54cm}
\usepackage{graphicx}
\usepackage{babel}
\begin{document}

\title{Comparison and analysis of ChIP-seq experiments}


\author{Colin Diesh}


\date{Senior Thesis Project, Fall 2012\\
$\,$\\
First semester final report - Outline}
\maketitle
\begin{abstract}
DNA-protein interactions are essential for gene regulation, and these
interactions can be detected by using chromatin immunoprecipitation
and high throughput sequencing (ChIP-seq). In particular, transcription
factor binding sites are identified as a significant enrichment of
the ChIP-seq data in particular DNA regions. The significance level
for finding these binding sites is somewhat arbitrarily, and in order
to compare data from different experiments, we want to normalize the
datasets to find shared significant binding sites. We use a normalized
difference score described by Zheng et al (2010) to find evidence
for shared binding sites from different ChIP-seq experiments. We also
evaluate other methods for finding conserved binding sites in ChIP-seq
data, and we look at techniques to correlate the transcription factor
binding sites with other genome features.
\end{abstract}

\section{Introduction}

The genes in our DNA are regulated by a variety of proteins that interact
with the DNA . Transcription factors in particular are proteins that
can effectively regulate DNA transcription by binding to the DNA.
These transcription factor binding sites can be detected by using
chromatin immunoprecipitation and high throughput sequencing (ChIP-seq)
by finding significant enrichment of the ChIP-seq data in particular
DNA regions. ChIP-seq uses special antibodies that crosslink the transcription
factors with DNA, and then high throughput sequencing is applied to
the DNA fragments from the ChIP method. The transcription factor binding
sites can be identified as peaks in the data -- a significant enrichment
of ChIP-seq reads over the background. The significance level for
finding these binding sites is somewhat arbitrarily, and in order
to compare data from different experiments, we want to normalize the
datasets to find shared significant binding sites. We used a normalized
difference score described by Zheng et al (2010) to find evidence
for shared binding sites from different ChIP-seq experiments. We also
evaluate methods for finding conserved binding sites in ChIP-seq data,
and we look at techniques to correlate the transcription factor binding
sites with other genome features.


\section{Background}

The platform we use for ChIP-seq is from Illumina and it uses some
proprietary technology for their Genome Analyzer and Solexa Sequencing.
In their process the library the DNA prep library is attached to adapters
and ``bridge PCR'' is used to build a cluster on the cell. Then
the 5' ends of the are sequenced and base-calls are done using image
analysis. Since many reactions are done in parallel we call this process
high-throughput sequencing. Then in the Illumina pipline reads are
aligned to the genome using a fast alignment algorithm called ELAND.
Once this work has been completed, then we can proceed to analyzing
the data to find transcription factor binding sites. 


\subsection{Model based analysis of ChIP-seq (Zhang et al. 2007)}

We did a source-code study of MACS, a popular free and open source
software program for analyzing ChIP-seq data, to find out details
about its implementation and processing. The published description
of the program is quite thorough, but we found additional info that
helped us understand the details of ChIP-seq processing. MACS is a
command line program that has a number of configurability options
and understanding the processing that MACS does is important for understanding
these options. 

MACS loads aligned ChIP-seq reads and it builds an internal representation
in a Python hash data structure called a dictionary. The dictionary
associates the chromosome as the key and the position and orientation
of the tags as the value. MACS does not use any other the details
about the sequence data or any mismatches in alignment for peak calling{*}.

After reading in the datafiles, MACS calculates a ``shift model''
to center the reads from each DNA strand. The shift model calculates
the offset between +/- strands to compensate the sequencing of the
5' ends of the fragments. To calculate the offset, a ``naive'' peak
finder that quickly scans the is used to find candidate peaks where
the number of tags in a scanning window is at least MFOLD greater
than the background. The background is calculated as the number of
reads we would expect in a window size given the genome size and total
number of reads.a minimum deviation what would be expected according
to a random distribution

\begin{lstlisting}
background=total_reads*bandwidth/gsize
min_tags=mfold*background
\end{lstlisting}


Then the algorithm uses the dictionary to grab all tags from each
chromosome. It loops over all tags and adds tags to a list if they
are within a inputted window size (bandwidth) and calls peaks when
the added tags exceeds min\_tags. Then using offset $d$ is calculated
as the median of the distance between the candidate peaks and all
reads are shifted by $d/2$ in the 3' direction.

To do peak finding, MACS evaluates all candidate peaks using the offset
corrected reads compared with the statistical distribution of the
background. The distribution is characterized as a Poisson with a
rate parameter calculated as the expected number of background reads
for global and local variations. Is simply the total number of reads.
The so-called background lambda is calculated as the expected number
of reads in a given scan window given the total number of reads and
the genome size. 

\begin{lstlisting}
lambda_bg0 = float(self.scan_window)*trackI.total/self.gsize 
\end{lstlisting}


Then the total number of tags to be called a peak is calculated as
the poisson CDF according to the likelihood of receiving a low probability
number of reads.Then all candidate peaks are processed to determine
if they pass the mark. The total number of reads in a given window
is compared with the rate parameter is also estimated for local windows
around the scanning window for 5k and 10k windows as variations in
the control data.

\begin{lstlisting}
window_size_4_lambda = max(peak_length,self.scan_window)                 
lambda_bg = lambda_bg0/self.scan_window*window_size_4_lambda*t_ratio 
local_lambda = max(lambda_bg,clambda_peak,clambda_lregion,clambda_sregion)
peak_pvalue = poisson_cdf(tlambda_peak,local_lambda,lower=False) 
if(peak_pvalue < self.pvalue)      
     final_peaks.append(peak)
\end{lstlisting}


{*}It is possible that analysis could reveal that mismatch alignments
somehow correlate with ChIP-seq data features

{*}{*} What is relationship between mfold background and Poisson background
(ie many less peaks called with large bandwidth compared with small
bandwidth!)

{*}{*}{*} Potentially make call graph or other diagram for informational
purposes


\subsection{Conservation of ChIP-seq binding sites}

While finding causes of binding variation is important for comparing
biological differences, the problem of finding conserved binding sites
is also important for measuring data integrity, reproducability, and
for assessing biological invariates. However, ChIP-seq experiments
are subject to many variables such as noise and bias, as well as normalization
issues (Shao et al. 2012). Therefore, developing a model for comparing
data similarity is important.

To find conserved binding sites, we want to identify a threshold for
peak finding that can be shared by both experiments. If we set a low
threshold, we may have to analyze many more binding sites, and potentially,
more false positives. In this situation, using the empirical false
discovery rate (FDR) to guide analysis is important. The P-values
reported by MACS do not necessarily tell us anything about control
data, however the FDR uses a data swap. Using FDR values less than
1\% is recommended for peak calling and conservation analysis (Bardet,
2012). This helps control the false positive rate from the MACS algorithm
since P-values do not assess the quality of control data and reads.
the quality of the peaks that MACS finds 

Other methods for identifying conserved binding sites have been developed
to model normalization. One application of note MANorm: a robust model
for quantitative comparison of ChIP-seq datasets\textit{ }(Shao et
al, 2012). MANorm normalizes the read counts using a linear regression
for the tag counts that are identified from shared binding sites.
(Figure 1). The normalization model is then extrapolated to all reads,
and the normalized read count is used to find conserved binding sites.
Additionally a bayesian Poisson model is used to find conserved binding
sites from both experiments concurrently. This effectively solves
the problem of using one experiment as prior information for identifying
peaks in another experiment.\\
\\
\includegraphics[scale=0.5]{\string"MAnorm_ a robust model for quantitative  - Zhen Shao\string".eps}

Figure 1. Workflow for identifying conserved peaks with MANorm (Shao
et al 2012)


\subsection{Genetic analysis of transcription factor binding variation in yeast
(Zheng et al. 2010)}

Zheng et al. analyzed the differences in ChIP-seq data for transcription
factor Ste12 from two yeast strains and different cross-strains between
them, and they found potential causes of transcription factor binding
variation. 

They first calculated a normalized difference score to get the ChIP-seq
'binding signal' (see methods section 2.1) for all experiments. Then
they analyzed the variation of NormDiff scores to identify 'variable
binding regions'. They identified cis- and trans- factors that contribute
to binding variation using QTL mapping to correlate 'variable binding
regions' with genetic markers. The cis-factors were identified as
genetic markers that are close to the variable binding regions whereas
the trans-factors were identified as genetic markers that are far
away from the binding sites that were still correlatedd with binding
variation (Figure 2)

\includegraphics[scale=0.7]{consensus3}\\
Figure 2. Modifications to DNA motifs in the yeast population correlated
with variable transcription factor binding (Zheng et al, 2010)\\
\\


In the QTL analysis, Zheng et al. also found trans-factors, defined
as markers that are located far from the variable binding regions
that contributed to binding differences. They aimed to identify trans-acting
genes responsible for these 'trans-QTL' and they used clustering algorithms
to validate binding differences in different subsets of the yeast
population with the genetic linkage (Figure 3). Then they identified
the binding differences that were closely associated with target genes
in two cases (FLO8 and AMN1). They tested for the effects different
genes on binding by using knockout methods in the parent strains and
found that they can be attributed to the binding differences in \\
\\
\includegraphics[scale=0.5]{\string"Figure 1\string".eps}

Figure 3. Hierarchical clustering of variable binding traits, and
association with markers used to validate the trans-factor associations
with binding variation (Zheng et al 2010)


\subsection{Tissue specific transcriptional regulation has diverged significantly
between human and mouse (Odom, Dowell et al. 2007)}

Odom, Dowell et al. used ChIP-chip tiling arrays to analyze binding
site differences in human and mouse hapetocytes (liver tissue). They
used orthologous genes in human and mouse, and they created tiling
arrays from the gene promoter regions and they proposed two approaches
for comparing binding sites from each species.\\


The summary of this paper and relevance to our work is pending.


\section{Previous work}

In our previous work, we sought to identify conserved binding sites
in order to improve ChIP-seq data comparison. We used data gathered
from two seperate yeast strains (S288c and YJM789) by Zheng et al
and a normalized difference score to get the background subtracted
ChIP-seq binding signal. We used MACS to find high confidence binding
sitesfrom both strains, and we used the normalized difference score
show evidence for shared binding sites in both experiments.


\subsection{Finding conserved binding sites}

We identified a set of high confidence binding sites in ChIP-seq data
using MACS (Zhang et al, 2008). However, we found that there was evidence
for shared binding sites from both parents at this threshold. We compared
the NormDiff scores from binding sites in one experiment with the
syntenic regions of the other experiments, and we found that there
was considerable overlap between the NormDiff scores of the shared
binding sites with the sites that were not called by MACS.


\subsection{Results}

We found 42 new binding sites in HS959 with that were conserved in
S96 that were not identified by MACS with P<0.05, and 4 of binding
sites that were conserved with P<0.01. We also found 76 binding sites
in S96 with P<0.05 that were conserved in HS959, and 13 of these binding
sites were conserved with P<0.01 

To evaluate the significance of these findings, we evaluated all binding
sites from S96 and identified 98\% (885/897) using our meethod at
P<0.05, and 60\% (563/897) binding sites at a P<0.01. In HS959 we
identified 85\% (729/824) of the same binding sites at P<0.05 and
46\% (365/824) at P<0.01\\
\includegraphics[scale=0.4]{Rplot25}\\
Figure 5. The maximum average NormDiff score is used to identify additional
binding sites that are conserved in each experiment. 


\section{Discussion}

We identified some additional evidence for conserved binding sites
that were not identified by other peak finding algorithms by using
our method. Our work could be extended to use more sophisticated statistical
models for NormDiff too. For example, we considered modifying NormDiff
scores to use data from multiple experiments in order to find conserved
binding sites. We also discuss some of the drawbacks of our method,
and the other applications of NormDiff.


\subsection{Modified normdiff score}

We propose using a modified NormDiff score that uses data from multiple
experiments. A NormDiff score that adds data from two experiments,
A1,B1,A2, and B2 can be defined as \\
\[
Z_{add}(x_{i})=\frac{(A_{1}-B_{1}/c_{1})+(A_{2}-B_{2}/c_{2})}{\sigma}
\]


Then the scaling factors $c_{1}$and $c_{2}$ are estimated from data,
and the variance is $\sigma=\sqrt{A_{1}+B_{1}/c_{1}^{2}+A_{2}+B_{2}/c{}_{2}^{2}}$ 

A NormDiff score that subtracts data from two experiments could also
be defined as
\[
Z_{sub}(x_{i})=\frac{(A_{1}-B_{1}/c_{1})-(A_{2}-B_{2}/c_{2})}{\sigma}
\]


The variance for $Z_{sub}$ would be the same as the variance for
$Z_{add}$. It is possible that adding and subtracting NormDiff scores
might be able to help in identifying conserved and differential binding.
For example, if the subtractive score is small but the additive score
is large, then the binding site is likely to be conserved in both
experiments.\\


\includegraphics[scale=0.4]{Rplot28}

Figure 6. $Z_{add}$ vs. $Z_{sub}$ for S96 and HS959. This plot is
similar to a MA plot because log product and log ratio are similar
to adding and subtracting the scores. The unique and shared binding
sites are colored

\includegraphics[scale=0.4]{Rplot05}

Figure 7. MA-plot of shared and unique peaks from S96 and HS959. Unique
peaks to S96 colored green, unique peaks to HS959 colored red, shared
peaks from both colored blue.

.


\subsection{Drawbacks of our approach}

Our approach tries to find evidence for conserved binding sites by
analyzing data using normalized difference scores. Our approach gives
some evidence of peaks with high confidence P-values, such as our
results of finding new peaks with P<0.01. However, these findings
have much less confidence than thresholds set by MACS (P<1e-5). Given
that our technique for calculating P-value is similar to MACS, by
finding large deviations from the background distribution, our results
are not very strong. 

One of the reasons for having less confidence in P-values is because
NormDiff uses control data for background subtraction. MACS on the
other hand uses only ChIP-seq data to calculate P-values, and does
not subtract control data. Instead MACS uses control data for filtering
and empirical FDR. Because of this, finding high confidence peaks
relies on using the FDR in MACS, and it is actually possible to use
FDR to adjust the P-value from MACS to be more leniant. So, because
the weakness of NormDiff to find conserved binding sites in ChIP-seq
data with confidence, it seems unproductive to do this analysis. 


\subsection{Applications of NormDiff}

NormDiff scores have interesting applications for finding conserved
and differential binding sites besides our method. One marked example
was the use of the NormDiff score to find variable binding regions
across many ChIP-seq experiments by Zheng et al. Another example is
what Zheng et al. describe as a ``transgressive score'' which is
used to evaluate whether binding sites from yeast cross strains were
inherited from their parents. The transgressive score analyzes the
variance of normdiff scores across all children compared with the
the difference of the parents: 
\[
transgressive=\frac{var(Z)}{Z_{parent1}-Z_{parent2}}
\]


They say that if the transgressive score is low, then the binding
site is ``inherited'' in a Mandelian fashion from one of the parents,
and so in some sense, it was conserved. However, if the transgressive
score is high, then the binding site is considered an extreme phenotype
that is transgressive. Their analysis shows in yeast that some even
lowly expressed peaks might be called transgressive because they are
not conserved with the same strength of the parents' binding sites. 

\includegraphics[scale=0.6]{untitled4}

Figure 8. ``ChIP-Seq signal tracks showing Ste12-binding sites that
segregate in a Mendelian (left) or transgressive (middle and right)
fashion. The colour indicates genotype background in the depicted
regions: red (S96), green (HS959). Asterisks indicate peaks of interest''
(Zheng et al. 2010)


\section{Proposed new direction}


\subsection{Problem statement}
\begin{enumerate}
\item Todo
\end{enumerate}

\section{Methods}


\subsection{Normalized difference scores}

The normalized difference (NormDiff) score is a useful statistic for
comparing ChIP-seq data that uses background subtraction and normalization
to obtain the ChIP-seq binding signal. The NormDiff score from Zheng
et al. uses a simple random model for ChIP-seq ($A$) and control
($B$) defined as

\begin{eqnarray*}
A & \sim & Poisson(f+g)\\
B & \sim & Poisson(cg)
\end{eqnarray*}


Where- 

$f$ represents the binding signal

$g$ represents the background noise

$c$ is a scaling factor between $A$ and $B$\\


Then, the NormDiff $Z$ is defined for each genome position $x_{i}$
as 

\[
Z(x_{i})=\frac{A(x_{i})-B(x_{i})/c}{\sigma}
\]
 

Then the scaling factor $c$ is estimated as the median ratio of $A/B$
and the variance $\sigma$ is estimated from the maximum variance
of $\sqrt{A+B/c^{2}}$ in a local window. Then, the background score
is where $Z~Normal(0,1)$ and deviations from the norm are caused
by the strong ChIP-seq binding signals. \\


\includegraphics[scale=0.3]{Rplot08}\includegraphics[scale=0.3]{Rplot05_2}

Figure 3. (left) Kernel density plot of NormDiff for whole genome
shows a approximately normal distribution of the background reads.
(right) Q-Q Plot shows that NormDiff scores that represent strong
ChIP-seq binding signals are highly different from the background.


\section{Conclusion}

Normalized difference scores have a variety of applications for comparing
ChIP-seq experiments. We used normalized difference scores to find
conserved binding sites in multiple experiments. However, there are
some significant challenges in this analysis. We will extend our methods
to classify binding sites according to their alignment similar to
Odom, Dowell et al and this will improve our ability to characterize
'variable binding regions' also. Overall, these methods will be able
to better identify binding site differences an similarities. 


\section{Bibliography}

\bibliographystyle{plain}
\nocite{*}
\bibliography{References}

\end{document}
