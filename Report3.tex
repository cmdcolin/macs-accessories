%% LyX 2.0.3 created this file.  For more info, see http://www.lyx.org/.
%% Do not edit unless you really know what you are doing.
\documentclass[english]{article}
\usepackage{mathptmx}
\usepackage{helvet}
\usepackage{courier}
\usepackage[T1]{fontenc}
\usepackage[latin9]{inputenc}
\usepackage{geometry}
\geometry{verbose,tmargin=2.54cm,bmargin=2.54cm,lmargin=2.54cm,rmargin=2.54cm}
\usepackage{graphicx}

\makeatletter

%%%%%%%%%%%%%%%%%%%%%%%%%%%%%% LyX specific LaTeX commands.
%% Because html converters don't know tabularnewline
\providecommand{\tabularnewline}{\\}

\makeatother

\usepackage{babel}
\begin{document}

\title{Comparison and analysis of transcription factor binding sites}


\author{Colin Diesh}


\date{Senior Thesis Project, Fall 2012\\
$\,$\\
First semester final report - Outline}
\maketitle
\begin{abstract}
Transcription factors are proteins which recognize and bind to short
DNA sequences, and they can play an important role in regulating DNA
transcription. Comparing transcription factor binding sites from different
experiments can show functional differences and similarities, however
binding sites have been shown to vary significantly even from closely
related individuals. Zheng et al. characterized transcription factor
binding variations by using a normalized difference score to find
'variable binding regions' across different experiments. We use the
normalized difference score from multiple experiments and we also
use the DNA alignment from different genomes in order to classify
binding sites. 
\end{abstract}

\section{Background}

Protein-DNA interactions play an important role in gene regulation,
and chromatin immunoprecipitation (ChIP) is a methodology for analyzing
these interactions. Antibodies that bind to transcription factors
or histone modifiers are used to immunoprecipitate the protein-DNA
complexes. Using chromatin immunoprecipitation and high throughput
sequencing (ChIP-seq) or oligonucleotide tiling arrays (ChIP-chip)
we can identify transcription factor binding sites using ``peak calling''.
Comparing ChIP data from multiple experiments is important to identify
functional similarities and differences. \\
\\
Here we will look at some background papers that identify binding
variations and also look at methods for finding conserved binding
sites. 


\subsection{Tissue specific transcriptional regulation has diverged significantly
between human and mouse (Odom, Dowell et al. 2007)}
\begin{enumerate}
\item Odom, Dowell et al. used ChIP-chip tiling arrays to analyze binding
site differences in human and mouse hapetocytes (liver tissue). 
\item They used orthologous genes in human and mouse, and they created tiling
arrays from the gene promoter regions
\item They proposed two approaches for data anlysis

\begin{enumerate}
\item - Gene centric approach for ChIP

\begin{enumerate}
\item Classify binding as true/false for each gene (tiling array as a whole)
\item Similar to Zheng et al. by identifying target genes 
\end{enumerate}
\item - Peak centric approach

\begin{enumerate}
\item Classify peaks within tiling array
\item Find 'peaks' using ChIP-chip and compare aligned sites
\item Peaks as conserved, turnover, gain/loss, or unaligned\\
\includegraphics[scale=0.75]{peak_classification}\\
Figure 1. Transcription factor binding sites are classified in terms
of DNA alignment (Odom, Dowell et al. 2007)\\

\item Odom, Dowell results many shared binding sites are not aligned (60\%+)\\
\\
\begin{tabular}{|c|c|c|c|c|}
\hline 
Shared binding sites & FOXA2  & HNF1A & HNF4A & HNF6\tabularnewline
\hline 
\hline 
Binding sites aligns & 34\% & 41\%  & 23\% & 30\%\tabularnewline
\hline 
Binding sites do not align & 39\% & 31\% & 38\%  & 41\%\tabularnewline
\hline 
No alignment & 28\% & 28\% & 38\%  & 30\%\tabularnewline
\hline 
\end{tabular}\\
\\
Table 1. Percent of binding sites that align from different transcription
factors in human and mouse hepatocytes
\end{enumerate}
\end{enumerate}
\item Questions about Odom, Dowell

\begin{enumerate}
\item How can we use their results to answer questions about more similar
functional comparisons. ChIP-seq where sequence comparison is important
for ChIP-seq sequence similarity?
\end{enumerate}
\end{enumerate}

\subsection{Genetic analysis of transcription factor binding variation in yeast
(Zheng et al. 2010)}
\begin{enumerate}
\item Zheng et al. analyzed binding sites from ChIP-seq experiments using
crosses from two yeast strains and found potential causes of transcription
factor binding variation

\begin{enumerate}
\item They first calculated the normalized difference score (NormDiff) for
each experiment to get the ChIP-seq 'binding signal' (see section
2.1) 
\item Then they analyzed the variation of NormDiff scores to identify 'variable
binding regions' across all experiments, and they identified cis-
and trans- factors that contribute to these variations

\begin{enumerate}
\item They used a 'single marker regression' to associate 'variable binding
regions' with genetic markers
\item They found cis-factors which are genetic markers that are close to
the variable binding regions that were identified as the main source
of binding variation (Figure 2)
\item They also found trans-factors, defined by 'variable binding regions'
associated with genetic markers far away from the binding sites

\begin{enumerate}
\item This analysis also identified trans-clusters which were variable binding
regions associated with a distinct class of genes
\item This analysis also tested for the trans-acting genes responsible for
the binding variation using knockout methods\\
\\
\includegraphics[scale=0.7]{consensus3}\\
Figure 2. Modifications to DNA motifs in the yeast population correlated
with variable transcription factor binding (Zheng et al, 2010)
\end{enumerate}
\end{enumerate}
\end{enumerate}
\end{enumerate}

\subsection{Conservation of ChIP-seq binding sites, techniques}

The problem of finding conserved binding sites is important for analyzing
ChIP-seq. ChIP-seq experiments are subject to many variables including
noise, bias, and normalization issues (Citation). To find shared binding
sites, we want to identify a threshold for peak finding that can be
shared by both experiments. If we set a low threshold, we may have
to analyze many more binding sites, and potentially, more false positives.
In this situation, using the empirical false discovery rate (FDR)
to guide analysis is important. The P-values reported by MACS do not
necessarily tell us anything about control data, however the FDR uses
a data swap. Therefore, using FDR values less than 1\% is recommended
for peak calling and conservation analysis (Bardet, 2012).

Other methods for identifying conserved binding sites have been developed
using a normalization model. One application of note MANorm: a robust
model for quantitative comparison of ChIP-seq datasets\textit{ }(Shao
et al, 2012). MANorm normalizes the read counts using a linear regression
for the tag counts that are identified from shared binding sites.
The normalization model is extrapolated to all peaks, and they use
the normalized read counts to find conserved binding sites. Additionally
a bayesian Poisson model is used to consider data from both experiments
concurrently.


\section{Method and approach}

We sought to identify conserved binding sites in order to optimize
the use ChIP-seq data gathered from two seperate yeast strains (S288c
and YJM789). We used a normalized difference score that was defined
by Zheng et al. to analyze the binding sites identified using MACS
from both strains, and we try to show evidence for shared binding
sites in both experiments.


\subsection{Normalized difference scores}

The normalized difference (NormDiff) score is a useful statistic for
comparing ChIP-seq data that uses background subtraction and normalization
to obtain the ChIP-seq binding signal. The NormDiff score from Zheng
et al. uses a simple random model for ChIP-seq ($A$) and control
($B$) defined as

\begin{eqnarray*}
A & \sim & Poisson(f+g)\\
B & \sim & Poisson(cg)
\end{eqnarray*}


Where- 

$f$ represents the binding signal

$g$ represents the background noise

$c$ is a scaling factor between $A$ and $B$\\


Then, the NormDiff $Z$ is defined for each genome position $x_{i}$
as 

\[
Z(x_{i})=\frac{A(x_{i})-B(x_{i})/c}{\sigma}
\]
 

Then the scaling factor $c$ is estimated as the median ratio of $A/B$
and the variance $\sigma$ is estimated from the maximum variance
of $\sqrt{A+B/c^{2}}$ in a local window. Then, the background score
is where $Z~Normal(0,1)$ and deviations from the norm are caused
by the strong ChIP-seq binding signals. \\


\includegraphics[scale=0.3]{Rplot08}\includegraphics[scale=0.3]{Rplot05_2}

Figure 3. (left) Kernel density plot of NormDiff for whole genome
shows a approximately normal distribution of the background reads.
(right) Q-Q Plot shows that NormDiff scores that represent strong
ChIP-seq binding signals are highly different from the background.


\subsection{Finding conserved binding sites}

We identified a set of high confidence binding sites in ChIP-seq data
using MACS (Zhang et al, 2008). However, we found that there was evidence
for shared binding sites from both parents at this threshold. We compared
the NormDiff scores from binding sites in one experiment with the
syntenic regions of the other experiments, and we found that there
was considerable overlap between the NormDiff scores of the shared
binding sites with the sites that were not called by MACS (Figure
4)


\subsection{Results}

We found 42 new binding sites in HS959 with that were conserved in
S96 that were not identified by MACS with P<0.05, and 4 of binding
sites that were conserved with P<0.01. We also found 76 binding sites
in S96 with P<0.05 that were conserved in HS959, and 13 of these binding
sites were conserved with P<0.01 

To evaluate the significance of these findings, we evaluated all binding
sites from S96 and identified 98\% (885/897) using our meethod at
P<0.05, and 60\% (563/897) binding sites at a P<0.01. In HS959 we
identified 85\% (729/824) of the same binding sites at P<0.05 and
46\% (365/824) at P<0.01\\
\includegraphics[scale=0.4]{Rplot25}\\
Figure 5. The maximum average NormDiff score is used to identify additional
binding sites that are conserved in each experiment. 


\section{Discussion}

We identified some additional evidence for conserved binding sites
that were not identified by other peak finding algorithms by using
our method. Our work could be extended to use more sophisticated statistical
models for NormDiff too. For example, we considered modifying NormDiff
scores to use data from multiple experiments in order to find conserved
binding sites. We also discuss some of the drawbacks of our method,
and the other applications of NormDiff.


\subsection{Modified normdiff score}

We propose using a modified NormDiff score that uses data from multiple
experiments. A NormDiff score that adds data from two experiments,
A1,B1,A2, and B2 can be defined as \\
\[
Z_{add}(x_{i})=\frac{(A_{1}-B_{1}/c_{1})+(A_{2}-B_{2}/c_{2})}{\sigma}
\]


Then the scaling factors $c_{1}$and $c_{2}$ are estimated from data,
and the variance is $\sigma=\sqrt{A_{1}+B_{1}/c_{1}^{2}+A_{2}+B_{2}/c{}_{2}^{2}}$ 

A NormDiff score that subtracts data from two experiments could also
be defined as
\[
Z_{sub}(x_{i})=\frac{(A_{1}-B_{1}/c_{1})-(A_{2}-B_{2}/c_{2})}{\sigma}
\]


The variance for $Z_{sub}$ would be the same as the variance for
$Z_{add}$. It is possible that adding and subtracting NormDiff scores
might be able to help in identifying conserved and differential binding.
For example, if the subtractive score is small but the additive score
is large, then the binding site is likely to be conserved in both
experiments.\\


\includegraphics[scale=0.4]{Rplot28}

Figure 6. $Z_{add}$ vs. $Z_{sub}$ for S96 and HS959. This plot is
similar to a MA plot because log product and log ratio are similar
to adding and subtracting the scores. The unique and shared binding
sites are colored

\includegraphics[scale=0.4]{Rplot05}

Figure 7. MA-plot of shared and unique peaks from S96 and HS959. Unique
peaks to S96 colored green, unique peaks to HS959 colored red, shared
peaks from both colored blue.

.


\subsection{Drawbacks of our approach}

Our approach tries to find evidence for conserved binding sites by
analyzing data using normalized difference scores. Our approach gives
some evidence of peaks with high confidence P-values, such as our
results of finding new peaks with P<0.01. However, these findings
have much less confidence than thresholds set by MACS (P<1e-5). Given
that our technique for calculating P-value is similar to MACS, by
finding large deviations from the background distribution, our results
are not very strong. 

One of the reasons for having less confidence in P-values is because
NormDiff uses control data for background subtraction. MACS on the
other hand uses only ChIP-seq data to calculate P-values, and does
not subtract control data. Instead MACS uses control data for filtering
and empirical FDR. Because of this, finding high confidence peaks
relies on using the FDR in MACS, and it is actually possible to use
FDR to adjust the P-value from MACS to be more leniant. So, because
the weakness of NormDiff to find conserved binding sites in ChIP-seq
data with confidence, it seems unproductive to do this analysis. 


\subsection{Applications of NormDiff}

NormDiff scores have interesting applications for finding conserved
and differential binding sites besides our method. One marked example
was the use of the NormDiff score to find variable binding regions
across many ChIP-seq experiments by Zheng et al. Another example is
what Zheng et al. describe as a ``transgressive score'' which is
used to evaluate whether binding sites from yeast cross strains were
inherited from their parents. The transgressive score analyzes the
variance of normdiff scores across all children compared with the
the difference of the parents: 
\[
transgressive=\frac{var(Z)}{Z_{parent1}-Z_{parent2}}
\]


They say that if the transgressive score is low, then the binding
site is ``inherited'' in a Mandelian fashion from one of the parents,
and so in some sense, it was conserved. However, if the transgressive
score is high, then the binding site is considered an extreme phenotype
that is transgressive. Their analysis shows in yeast that some even
lowly expressed peaks might be called transgressive because they are
not conserved with the same strength of the parents' binding sites. 

\includegraphics[scale=0.6]{untitled4}

Figure 8. ``ChIP-Seq signal tracks showing Ste12-binding sites that
segregate in a Mendelian (left) or transgressive (middle and right)
fashion. The colour indicates genotype background in the depicted
regions: red (S96), green (HS959). Asterisks indicate peaks of interest''
(Zheng et al. 2010)\\



\section{Proposed new direction}


\subsection{Problem statement}
\begin{enumerate}
\item Todo
\end{enumerate}

\section{Conclusion}

Normalized difference scores have a variety of applications for comparing
ChIP-seq experiments. We used normalized difference scores to find
conserved binding sites in multiple experiments. However, there are
some significant challenges in this analysis. We will extend our methods
to classify binding sites according to their alignment similar to
Odom, Dowell et al and this will improve our ability to characterize
'variable binding regions' also. Overall, these methods will be able
to better identify binding site differences an similarities. 
\end{document}
